\includegraphics[page=2,scale=0.25,trim={0 7cm 0 0},clip]{slides/learningmealy.pdf}
\includegraphics[page=3,scale=0.25,trim={0 7cm 0 0},clip]{slides/learningmealy.pdf}
\includegraphics[page=5,scale=0.25,trim={0 5cm 0 0},clip]{slides/learningmealy.pdf}
\includegraphics[page=7,scale=0.25,trim={0 10cm 0 0},clip]{slides/learningmealy.pdf}

\subsection{Introduction}

\includegraphics[page=21,scale=0.25,trim={0 5cm 0 0},clip]{slides/learningmealy.pdf}

\subsection{Active Automata Learning primer}

\subsection{Mealy Machines}
A Mealy machine is defined as a tuple $M = \langle S, s, \Sigma, \Omega, \delta, \lambda\rangle$ where
\begin{itemize*}
	\item S is a finite nonempty set of states (be n = |S| the size of the Mealy machine),
	\item s $\in$ S is the initial state,
	\item $\Sigma$ is a finite input alphabet,
	\item $\Omega$ is a finite output alphabet,
	\item $\delta$: Sx$\Sigma\rightarrow$S is the transition function, and
	\item $\lambda$: Sx$\Sigma\rightarrow\Omega$ is the output function.
\end{itemize*}

Extend $\delta$ e $\lambda$ for words\\

\begin{tabular}{cl}
	$\delta*$ & $: Sx\Sigma * \rightarrow S$: \\ 
	& $\delta* (s,\epsilon)= s$\\ 
	& $\delta* (s, \alpha w) = \delta * (\delta(s,  \alpha), w)$ 
\end{tabular}

\begin{tabular}{cl}	
	$\lambda*$ & $: Sx\Sigma * \rightarrow\Omega$:\\
	& $\lambda* (s,\epsilon)=\emptyset$\\
	& $\lambda* (s, w\alpha) = \lambda(\delta * (s, w), \alpha)$
\end{tabular} 

\subsection{Nerode relation}

Let $P:\Sigma*\Omega$ {\tiny (let’s define a Mealy machine just as a function that, given a word, provides the single output observed after having that word in input)}\\


Definition (Equivalence of words w.r.t. $PL$)


Two words $u,u' \in \Sigma*$ are equivalent w.r.t. $\equiv P$, if and only if for all continuations $v \in \Sigma*$, the concatenated words $uv$ and $u'v$ are mapped to the same output by $P$:

\begin{center}
	$u \equiv_{P} \Leftrightarrow u'( \forall v \in \Sigma^*. P(uv) = P(u'v))$ %chktex 48
\end{center}

The idea is that each equivalence class (over $\equiv_{P}$) represents a state in the hypothesis. Agree?

\subsection{Counterexample handling}

\subsection{Other formalisms}

\subsection{Towards Nondeterminism}